% intro chatper to C learning book

\chapter{Introduction}

\paragraph{   } C is a low-level programming language designed for high portability and fast performance. C first originated in the late 1970's when Brian Kernighan and Dennis Ritchie wrote the book, \textit{The C Programming Language}. C has tremendously influenced the world of programming, leading to the development of C++, an object oriented version of C, and the syntax of many other languages, like Java. C is still widely used today, in many applications such as databases, kernels, drivers, and much more. One of the most popular programming languages, Python, runs on an interpreter written in C.

\par This chapter will introduce the very fundamental ideas behind programming, and work as a primer for learning actual C concepts. Ideas such as addressing, memory, the stack, the heap, variables, printing, files will be discussed with friendly examples. If you have no or little experience in programming, it is highly recommended you start with this chapter. Otherwise, skip ahead to the next chapter.

\section{Programs and Computing}

\par In the most basic sense, a computer is a machine which can complete tasks by running instructions commonly known as "code". A computer cannot read or understand human languages, like English. A computer only understands instructions in a format known as \textit{binary} code. Binary code is composed entirely of one's and zeros. $00000001$ is an example of binary, which actually translates to the number, $1$. 
\paragraph{ A programming language} is a human readable language which is \textit{compiled} into binary code. Such languages serve as a intermediate way of communication between a computer's mechanics and a human mind's mechanics. These languages allow humans to instruct computers to perform almost any task imaginable. A program is a file or series of files of code written in a particular programming language. Web browsers, mobile phone applications, websites, are all programs. All of them are in some way compiled into binary code, which is then executed by a computer as instructions. Programs that need high performance are often written in C or C++, a tightly related language to C.
% layer language diagram
\begin{figure}[h]
  \begin{center}
  \begin{tikzcd}
     &Human\hspace{2pt} Language \arrow[d] \\
     &Programming\hspace{2pt} Language \arrow[d] \\
     &Binary\hspace{2pt} Language
   \end {tikzcd}
   \end{center}
\caption{The layers of communication between humans and computers}
\end{figure}

\par The format in which humans convey information through language is very different than how computers convey information. Programming languages are not at all similar to learning a foreign language. The \textit{grammar} of programming languages is far more restricted than that of a spoken language. When two people speak to one another, they are conveying information that is often \textit{declarative} in nature, where information is stated in some format as true. "I cook food," is a declarative sentence, it states data or information, it does not detail commands or instructions to follow.
\paragraph{Imperative programming} involves the communication and passing of information as commands and instructions. C is an imperative programming language, so is Java, and C++. Such languages focus on grouping commands into executable sets of instructions to compute larger more complex values. In many imperative languages, groupings of commands are called \textit{functions}. In the most basic sense, functions apply commands on some specified set of values.
\paragraph{A function} is a special set of commands written in code, with some specified input values and output values. Input values to functions are usually called parameters. Functions are used in programming languages to tell the computer to perform specific instructions onto some set of parameters. They generally take the form of:

$$
function:\hspace{2pt} name(input) \rightarrow output
$$

The input can be on or more values, so can the output. Functions will be explored more in later chapters. \\

% list of definitions discussed:
\emph{\textbf{Definitions}}
\begin{enumerate}
\item \textbf{Declarative programming}: A style of programming where data is stated and declared.
\item \textbf{Imperative programming}: A style of programming that uses commands and instructions that compute data.
\item \textbf{Functions}: A group of commands executed on some specified set of values.
\end{enumerate}

\section{Memory}

\paragraph{   } When a person performs day to day tasks, like driving to work, making meals, or writing letters, they use their memory of past events. Even though the brain is not a computer, it still records data from events we experience which we think about or remember at some later time. When a computer executes binary code or compiled code from a programming language, it needs to use memory. If humans had no way of remembering or recalling anything in the past, they could never solve any problem, mathematical or otherwise. Yet, a computer's memory is not truly chronological. Consider the following example:

\begin{center}
\emph{Tom needs to buy groceries to make a soup. He knows he needs to buy tomatoes and corn. He makes a list, specifying one tomato and two cobs of corn. He goes tot he grocery store and checks his list, to buy the items he wanted.}
\end{center}

In this example, \textit{Tom} needs to buy some amount of vegetables, so he uses a list to check which items he should buy. If a computer were to purchase or order groceries, it would also need a list of what items to purchase. In the previous section, we discussed functions, an arrangement of instructions that are executed on some input values. Lets look at an abstract example.

$$
function:\hspace{2pt} BuyGroceries(list) \rightarrow PurchasedItems
$$

Here, a function is taking in an input of a list, and outputting the items the list stated to be purchased. The output is related to what the list holds in terms of data. More specifically:

% arrow diagram for I/O of a general function
\begin{center}
\begin{tikzcd}
 &List \arrow[dash]{d} \arrow[r] &PurchasedItems \arrow[dash]{d} \\
 &input &output
\end{tikzcd}
\end{center}

To keep track of the list, when a computer is executing commands, it needs to store that list. Secondly, it needs a place to output those purchased items to. To store such information, for a simple function as shown or many other tasks, a computer uses memory.

\subsection{What is memory?}

\paragraph{   } Memory is a physical component inside a computer that stores temporary data and information necessary for a computer to complete tasks. When a program receives an input, such as a mouse click on a button, or message of text, it needs to have a place to store that input, and any other values it needs to compute the desired output. To start with, we can think of memory as a finite number of boxes, each which can store a value. Such boxes allow us to reference that value in order to change it or create new values. Here is an example: \\

\emph{\textbf{Example:}}
% example picture of three boxes of memory.
\begin{center}
$$
A:[0] \hspace{3pt} B:[0] \hspace{3pt} C:[0]
$$

\emph{Let A, B, and C be boxes of memory. Initially, they all hold the value 0. A memory box must always hold some value. Next, let us inset or load numbers into boxes B and C.}

\begin{tikzcd}
& &2 \arrow[d, "load"] &3 \arrow[d, "load"] \\
&A:[0] &B:[2] &C:[3]
\end{tikzcd}

\emph{Memory box B and C can be added together, with the result of the mathematical operation being stored in box A. }

\begin{tikzcd}
&B \arrow[rd] &  &C \arrow[ld] \\
& &A:[5] &
\end{tikzcd}

\emph{The final state of each memory box is :}

$$
A:[5] \hspace{3pt} B:[2] \hspace{3pt} C:[3]
$$
\end{center}

In this example, at every step, the memory of what value A, B, or C holds is persisted. Those values are stored and changed as the commands continue to be executed. This is an example of how memory in registers also works, a topic that will be covered later on.

\subsubsection{Types of memory}

In a computer, there are a few different types of memory. They are used for different tasks, and different roles. Some are also slower than others.

\begin{enumerate}
\item \textbf{Registers}: This is the fastest, most highly accessible form of memory, it's commonly use in executing instructions.
\item \textbf{RAM}: Also known as simply memory or local memory, this is memory used to store temporary values during the time a program is running. In most programming languages assignment statements, such as $a = 3$ are using local memory.
\item \textbf{Disk}: This is where a computer stores files, the least accessible, yet longest lasting form of memory. 
\end{enumerate}

Memory is measured in units. The smallest, accessible unit of memory is called a byte. Computers and files often measure data in \textit{kilobytes} or \textit{megabytes}. These mean a thousand bytes and a million bytes respectively. The number system behind memory and data will be discussed in the next section.

\section{Numbers}

A computer and it's programs use numbers, just as humans encounter and use numbers in their daily lives. Phone numbers, street addresses, freeway exits, numbers exist all around us. Numbers are the core component of any data inside a computer. Computers can only understand sequences of numbers, they don't see text or images like living beings can. In a previous section, we discussed \textit{binary} language in reference to communicating with a computer. Binary language is just a composed sequence of ones and zeros. This section will provide insight into the system and types of numbers used in programming, such as binary numbers.

\subsection{Binary Numbers}

\paragraph{Numbers} are quantifiable, distinct values that exist in different sizes, composed of digits. The possible digits a number can have are determined by it's radix. The size of a number is determined by the \textit{domain} of values it is assigned to represent. To better understand this, let's investigate the system of binary numbers.

% Binary number example
\emph{\textbf{Example: Binary Numbers}}
\begin{center}
\emph{A binary number, $B^l_r$ has a radix, $r$ and a length, $l$. The radix for all binary numbers is 2, while the length is variable depending on the value represented. The Binary number for 2 is $10$. The beginning binary numbers are:}
\begin{align*}
0 &\equiv 0 \\
1 &\equiv 1 \\
2 &\equiv 10 \\
3 &\equiv 11 \\
4 &\equiv 100 \\
5 &\equiv 101 \\
6 &\equiv 101
\end{align*}
\emph{For every position in a binary number, there are only two possible values, 1 or 0. As the integer represented in the binary number becomes larger, so does the length of the binary number.}
\end{center}

\par Every 1 or 0 in a binary number is called a \textit{bit}. You may be familiar with the term 64-bit or 32-bit operating systems. This size refers to the size of the binary number that holds instructions for computers with such systems to execute. Binary numbers are usually divided by increasing powers of two, such as 8-bit, 16-bit, 32-bit, and 64-bit numbers. Each class of a binary number, can only hold a maximum number of values. Also called permutations, the number of permutations for a binary number is given by the formula:

$$
B = 2^n
$$

where n a finite length, and 2 is the radix of binary numbers. In the case of 8-bit numbers, 

$$
B_8 = 2^8 \equiv 256
$$

8-bit numbers, also called \textit{bytes}, can hold the number values between 0 and 255. From the 8-bit number $00000000$ to $11111111$, there exists 256 possible permutations. We will learn in later chapters that bytes the smallest addressable unit of memory. For now, we can think of a byte as the smallest number to represent a box of memory.

\subsection{Integers}

\paragraph{   }In contrast to binary numbers, integers are numbers you are likely largely familiar with from everyday life. $100$, $500$, $6,000,000$ are all integers. An integer is any positive or negative whole number. Numbers with decimal portions are called \textit{float} numbers and will be discussed later on.

\par Integers can be represented in many sizes, but in most programming languages, they are represented by 32-bit or 64-bit numbers. For a 32-bit number, the maximum value is:
% max val for integer
$$
N_{32} = 2^{32} \equiv 2,147,483,647
$$

for a 64-bit number, the maximum value is
% Max 64-bit number
$$
N_{64} = 2^{64} \equiv 9,223,372,036,854,775,807
$$

\noindent\rule{4cm}{0.4pt} \\
\emph{Note: You may have noticed that 2 is also used above as a radix for integers. As mentioned previously, a computer can only interpret and understand binary language. Integers in the exact form humans read them are not binary. Thus, each integer in a programming language is mapped to a binary number. } \\
\noindent\rule{4cm}{0.4pt}

Integers can also be \textit{signed} or \textit{unsigned}. Being signed or unsigned allows us to split the number of available values a sized number can represent. The most common use case is to represent both positive and negative numbers. \\

\emph{\textbf{Definitions}}
\begin{enumerate}
\item \textbf{Binary Number}: A number of some length composed entirely of 1's and 0's.
\item \textbf{Integer}: A whole, positive or negative real number, such as $55$.
\item \textbf{Radix}: The number of possible digits a number can be composed from.
\item \textbf{Permutation}: An ordering of elements, such as digits, where the order and presence matter, e.g $101$ and $110$.
\end{enumerate}

\section{Compiling a Program}

\paragraph{   } The C programming language is a compiled programming language, meaning the textual representation of the language as a whole is converted into binary or \textit{machine} language at once. The result of that conversion is called an executable. Executables are programs that can be run by a computer. In order to transform a C program into an executable a compiler needs to be used. A compiler is a program that takes an input of one or more C files, and returns or results in an output of an executable.

\par Nearly all computers come with a default C compiler. The most common and widely used is called \textit{gcc}, a compiler that is available for windows, macintosh and linux operating systems. Alternatively, many websites offer online compilers that you can write and compile code on web page. That will only display a textual output of the program, but that's more than sufficient for the course of this book.

\par To compile a C program locally on a computer with a \textit{UNIX} operating system, such as macintosh or linux, do the following. We will compile and run the following program as an example

%hello world C example
\begin{lstlisting}[style = customc]
#include <stdio.h>

int main(void)
{
      puts("Hello World!");
      return 0;
}

\end{lstlisting}

\textbf{Steps:}
\begin{enumerate}
\item Open your \textit{Terminal} program.
\item Type the command emph{vi program.c}. A new view will open in your terminal.
\item Press I to enter insert mode, and paste the above program into the terminal.
\item Press the \textit{esc} key, then type \emph{:wq}, to write and quit the text program.
\item Type the following command \emph{gcc program.c -o program}
\item Then finally, run the program by typing \emph{./program.c}
\end{enumerate}

You should get the following response: \\


\begin{lstlisting}[style = customc]
Hello World!
\end{lstlisting}

This program is a basic, fundamental piece of code that tells the computer to \textit{put} the text "Hello World!" into your terminal. Next, the specific portions and statements in this program will be discussed.

\subsection{Structure of a Program}

\paragraph{   } A C program before it's compiled into an executable, is a collection of one or more \emph{.c} files. Amongst these files, is a function called \textit{main}. The main function of a C program contains the entry point, the starting point for the instructions and commands the program will perform. You can think of a program as an entrance for the program to accept input. 

\begin{figure}[h]
\begin{center}
    \begin{tikzcd}
    & &User \arrow[d] & \\
    &  &main() \arrow[ld] \arrow[rd] & \\
    &func1() & &func2()
    \end{tikzcd}
\end{center}
\caption{ A diagram of a program with a main function.}
\end{figure}